\chapter{Introduction}
\label{ch:1}

What are the three most important factors in selling real estate? Location, location, and location. This is applicable not only in real estate but also in retail business \cite{anderson1997location}. Deciding where to locate business has always been a problem that people continuously try to solve worldwide. Throughout the time, most retailers would make a decision based on personal experience and instinct, regarding the process very much as an ``art``. People would mainly use very subjective techniques. Some of them are no more than ``hunches`` based upon experience \cite{hernandez2000art}.

In the retail environment, businesses are surrounded by an enormous amount of data and variables that can influence their success. As information systems evolved, research procedures became more sophisticated. For retailers, this presented a challenge: without using location decision procedures to improve objectivity, they risked falling behind businesses that adopted such methodologies \cite{hernandez2000art}. Retailers must carefully select and coordinate these tools to ensure they complement each other and provide a comprehensive view of the decision at hand. Otherwise, they risk making false decisions or mistakes.

A solution is to build a system to aid retailers in making informed location decisions. Such a system could utilize one of the procedures, which are designed to assist retailers in the decision-making process, particularly in identifying optimal business locations. This thesis adopts one notable methodology outlined in the journal \cite{roig2013retail}. This procedure enables users to analyze multiple datasets, utilize GIS features for location selection, and input their preferences into the system, which makes the procedure flexible and suitable for every retailer who chooses to utilize it \cite{roig2013retail}. Such a system should minimize the amount of work that needs to be done by retailers to analyse and locate the best possible site based on the provided data in any region.

Chapter \ref{ch:2} dives deep into the theory of retail site location assessment procedure. It is necessary to have a clear understanding in order to build a solid system. Chapter \ref{ch:3} introduces \textit{Geographic Information System} (GIS), its methods and the technologies behind it because the final solution partially implements them. Chapter \ref{ch:4} describes target audience for the system, its use cases and functional requirements. It also dives into existing solutions that are available on the internet. Chapter \ref{ch:5} focuses on design, outlining the visual and functional aspects of the system based on the requirements established in chapter \ref{ch:4}. Chapter \ref{ch:6} dives into the implementation of the system, detailing the technologies employed, design patterns utilized, and the steps taken to achieve the desired outcome. Chapter \ref{ch:7} outlines the testing procedures applied on the system, strategies for mitigating performance issues, and guidance on utilizing the system with real-world data.